
\documentclass[12 points, a4paper]{article} %
\usepackage{graphicx,amssymb} %
\textwidth=15cm \hoffset=-1.2cm %
\textheight=25cm \voffset=-2cm %
\renewcommand{\rmdefault}{ptm}
\pagestyle{empty} %
\date{\today} %

\def\keywords#1{\begin{center}{\bf Keywords}\\{#1}\end{center}} %
% Please, do not change any of the above lines

\begin{document}

% Type down your paper title
\title{Emotion detection to upgrade Customer Experience}

% Authors
%\author{Author 1, Author 2, Author 3 \\ %
%       University of Hamburg (Germany) \\ \\ % Affiliation 1
%       Author 4 \\ % If any other author with different Affilation
 %      University of Author 4 (Country) \\
\author{
Abisheck Kathirvel (CB.EN.U4CSE18404) \\
EVK Praneeth (CB.EN.U4CSE18419) \\
Rohith Rajesh (CB.EN.U4CSE18450) \\
Sanjith Ragul V (CB.EN.U4CSE18453) \\
\\ % Affiliation 2 (if needed)
       % New author \\
       % New affiliation \\
       % Add authors and affiliation as needed
Project Mentor: Mr. Sabarish B.A, Asst. Prof,\\
Dept. of Computer Science \& Engineering\\
Amrita School of Engineering\\
Amrita Vishwa Vidyapeetham
       %\tt{author@university.com} % Only one corresponding e-mail
       }%
\maketitle
\thispagestyle{empty}
% The abstract
\begin{abstract}
Growing technological development business platforms have shifted to web and cloud-based environments. Every customer needs privacy with respect to the purchases and services they seek. All business centers provide customer service support to clarify their queries in services or products. The present feedback system uses QA which ends in a biased feedback that leads to inaccurate customer experience. Interaction between the customer and executive gets affected by various reasons including inappropriate executive, extended wait time, etc. This project provides a framework that identifies the mood/emotion of the customer based on the initial chat query in a chat application and then uses sentiment analysis and routes the request to the appropriate technical expert to solve the issue which in turn improves the customer experience. After the call is established, using speech recognition the system identifies the emotion of the customer for the clarification/service provided by the person and grades them automatically. This framework for improving customer experience through sentiment analysis and emotion detection involves two phases: emotion-based call routing and auto-grading of service professionals.

The project proposes a Convolutional Neural Network(CNN) based architecture for Speech Emotion
Recognition and classifies speech into angry, neutral, disappointed or happy. The outcome of this proposed system is to upgrade the customer experience by analyzing the calls at the customer care center. This system can also be used in various fields other than feedback systems like in the diagnosis of physiological disorders and counseling as emotion is an important topic in psychology and neuroscience.\

\end{abstract}
\keywords{Sentiment analysis, Emotion detection, Machine learning, \\  NLP, Customer service, Speech recognition} % Write down at least 3 Keywords
% \section{Introduction}
\end{document}